\documentclass{beamer}
\usepackage[punct=kaiming, fontset=founder]{ctex}
\usepackage{parskip}
\usepackage{hologo}
\usepackage{calc}
\usepackage{hyperref}
\usepackage{minted}
\usepackage{ifxetex}
\newcommand{\TitanSnow}{Titan\makebox[0.1em]{\ }Snow}
\title{\LaTeX{} environment setup}
\subtitle{Special for Chinese}
\author{\TitanSnow}
\date{\today}
\begin{document}
    \frame{\titlepage}
    \begin{frame}
        \frametitle{What's this?}
        Just a \textit{beamer} test :)
    \end{frame}
    \begin{frame}
        \frametitle{Why this?}
        在配置\LaTeX{}的过程中,
        遇到了不少坑。
        有很多人使用错误或过时的方法,
        还广为流传,
        尤其是在中文处理方面。
        这份材料主要介绍了一些经验和推荐的方法,
        以供参考。
    \end{frame}
    \begin{frame}
        \frametitle{Why \LaTeX{}?}
        Microsoft Word \textit{versus} \LaTeX\pause

        笑话,你拿Word跟\LaTeX{}比也太自不量力了。\pause

        可配置的标点压缩你试试?
        \begin{description}
            \item[全角式]{\rmfamily\ctexset{punct=quanjiao}客喜而笑,洗盏更酌。肴核既尽,杯盘狼藉。相与枕藉乎舟中,不知东方之既白。}
            \item[半角式]{\rmfamily\ctexset{punct=banjiao}客喜而笑,洗盏更酌。肴核既尽,杯盘狼藉。相与枕藉乎舟中,不知东方之既白。}
            \item[开明式]{\rmfamily\ctexset{punct=kaiming}客喜而笑,洗盏更酌。肴核既尽,杯盘狼藉。相与枕藉乎舟中,不知东方之既白。}
        \end{description}\pause
        我发现开明式的标点压缩有利于增强句子的整体性,
        不会使子句与子句之间支离破碎,
        更利于阅读。(你用西文逗号当我没说)
    \end{frame}
    \begin{frame}
        \frametitle{Why \LaTeX{}?}
        \begin{description}[labelwidth=\widthof{Disadvantage}]
            \item<2->[Advantage]{
                \begin{itemize}
                    \item<2-> 动态规划排版算法
                    \item<3-> 图灵完备
                    \item<4-> 数学表示体系完备
                    \item<5-> 文本源码
                    \item<6-> \textellipsis
                \end{itemize}
            }
            \item<7->[Disadvantage]{
                \begin{itemize}
                    \item<7-> 编译贼jb慢
                \end{itemize}
            }
        \end{description}\pause[8]

        \hologo{LuaLaTeX} with \hologo{LuaTeX}-ja,
        编译速度那酸爽,用过都知道。
    \end{frame}
    \begin{frame}
        \frametitle{Why \LaTeX{}?}
        Microsoft Powerpoint \textit{versus} \LaTeX\pause

        这个没法比\textellipsis
        毕竟人家是专门做ppt的。
        pdf是没有动画的,
        这点就比不了了。

        没有动画的演示文稿当然是可以做的,
        就像本文一样,\LaTeX{} with \textit{beamer}。
    \end{frame}
    \begin{frame}
        \frametitle{Donald E. Knuth}
        \TeX 的 author

        \begin{itemize}
            \item<2-> Computational complexity of algorithms
            \item<3-> Dancing Links
            \item<4-> Knuth-Morris-Pratt algorithm
            \item<5-> \textellipsis
        \end{itemize}\pause[6]

        想必你们已经知道是谁了。
    \end{frame}
    \begin{frame}
        \frametitle{What are \TeX{}, \LaTeX{}, \hologo{pdfLaTeX}, \hologo{XeLaTeX}, \hologo{LuaLaTeX}\textellipsis?}
        就像认为Python就是CPython一样,
        很多人认为\LaTeX{}就是latex那个引擎(恕我表达不清楚)。
        woc,那玩意现在还有人用吗?

        \LaTeX{}是一种语言(姑且这么讲),
        有很多种实现。
        就像Python之于CPython和PyPy。
        一般有三种常用的引擎:\pause

        \begin{description}[labelwidth=\widthof{\hologo{LuaLaTeX} \textsubscript{with \hologo{LuaTeX}-ja}}]
            \item<2->[\hologo{pdfLaTeX}] 用于西文不使用额外字体排版
            \item<3->[\hologo{XeLaTeX} \textsubscript{with \hologo{Xe}CJK}] 用于中文排版
            \item<4->[\hologo{LuaLaTeX} \textsubscript{with \hologo{LuaTeX}-ja}] 用于日文排版
        \end{description}\pause[5]

        \TeX{}就不用管了,我不会。
    \end{frame}
    \begin{frame}
        \frametitle{\TeX{}Live Installation}
        终于开始讲正题了,激不激动?\pause

        力荐\TeX{}Live。其它发行版?不存在的。\pause

        不建议使用系统包管理器(如apt)。
        系统包管理器里的\TeX{}Live一般不是最新的。\pause

        Steps:
        \begin{enumerate}
            \item<4-> Download \href{http://tug.org/texlive/acquire-netinstall.html}{\TeX{}Live installer}
            \item<5-> Install it
            \item<6-> Have fun!
        \end{enumerate}
    \end{frame}
    \begin{frame}[fragile]
        \frametitle{Hello, \LaTeX{}!}
        \begin{minted}{latex}
\documentclass{article}
\begin{document}
    Hello, \LaTeX{}!
\end{document}
        \end{minted}

        Use \hologo{pdfLaTeX}, \hologo{XeLaTeX} or \hologo{LuaLaTeX} to compile.
    \end{frame}
    \begin{frame}[fragile]
        \frametitle{你好,\LaTeX{}!}
        \begin{minted}{latex}
\documentclass{ctexart}
\begin{document}
    你好,\LaTeX{}!
\end{document}
        \end{minted}

        Use \hologo{XeLaTeX} or \hologo{LuaLaTeX} to compile.
    \end{frame}
    \begin{frame}[fragile]
        \frametitle{こんにちは \LaTeX{}!}
        \begin{minted}{latex}
\documentclass{ltjsarticle}
\begin{document}
    こんにちは \LaTeX{}!
\end{document}
        \end{minted}

        Use \hologo{LuaLaTeX} to compile.
    \end{frame}
    \begin{frame}
        \frametitle{编辑器}
        \begin{itemize}
            \item<2-> \TeX{}studio
            \item<3-> Vim
        \end{itemize}
    \end{frame}
    \begin{frame}
        \frametitle{The End}
        \centering
        想必大家都会\LaTeX{},顺着本文环境也配好了。\pause

        \Huge\textbackslash{}bye\pause

        \small\ifxetex Gen by \hologo{XeLaTeX} with \CTeX\fi
    \end{frame}
    \begin{frame}
        \Huge
        \[
            e = \lim_{n \rightarrow \infty} \left(1+\frac{1}{n}\right)^n
        \]
    \end{frame}
\end{document}
