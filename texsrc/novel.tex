\documentclass[a5paper, punct=kaiming, fontset=none]{ctexart}
\setmainfont{NotoSansSC-DemiLight.otf}
\setCJKmainfont{NotoSansSC-DemiLight.otf}
\setlength{\parskip}{1ex}
\newcommand{\emptypar}{\ }
\begin{document}
“在下安娜斯塔西娅·托里斯提加,来自北原。大家可以叫我斯提娅。初次见面,请多指教。”斯提娅慢慢地弯下腰,正好避开了讲台下射来的目光。很好,没有出错,没有一点感情,斯提娅脸上没有任何表情。

“北原?那个城邦不是……”

“托里斯……有这个姓吗?”

……

斯提娅径直穿过了窃窃私语。是的,她一点也不在意,如同这些话语的对象是另一个人,亦或是坐在课桌前的实体并不存在。直到她无声地坐下,班主任嘉奈老师擦去了黑板上她的名字,开始介绍下一个同学,窃窃私语才停止,就如同什么也没有发生过,她从未出现过。

与预期没有偏差,斯提娅似乎是躲过一劫的感受。作为一个从边境城邦来的伪异能者,斯提娅只希望自己如从前一样,一个人蜷缩在墙角,不要有人关注。教堂里的日子多令人怀念,自己就是一个普通的孤儿,众多孤儿中的一员,既不能靠铁拳在其中拉帮结派,又无能凭交际多获得晚餐的一片面包。一个普通的孤儿,将来大概会是一个洗衣工,挣取底层人平均的报酬,换取足以果腹的食物,过大多数人的一生。没有人会知道,也没有人会尝试去了解。在沙滩上一个脚印也不会留下。

人生轨道的第一次偏差,是火焰一般的明艳。当教堂毁于烈焰,她从图书塔里被救出,幸存者祈祷着,祈求神明惩罚异能者的罪恶,保全他们的生命和财产。斯提娅手里攥着半截被烧毁的书,明白自己已经无法回到原先的轨道上。

“托里斯……同学?”就如一双大手将斯提娅的思绪攥起,斯提娅从回忆中回过神来,急急忙忙地偏过头,掩藏慌张的目光。“托里斯提加同学!”

斯提娅不知道如何回应。俯身在她面前的是一张笑脸。斯提娅见过太多的笑脸了,有小教堂的神父的笑,有街边乞讨者的笑,有集市上马车里贵妇人的笑,斯提娅却无法分辨现在她所见的是那种笑脸,只觉得这笑脸似乎是可以信赖的,似乎与之前所有所见的笑脸皆不同,尤其是那双眼睛,深棕色的瞳孔映射出的不是压抑混沌的光芒,却是如镜般晶亮,正和斯提娅自己碧绿却暗淡的瞳孔相反。

斯提娅的经验警告她,笑脸都是别有目的的,此时最好赶紧披上外衣。然而,此刻斯提娅却压抑不住想了解笑脸主人的冲动。那是修长水手服包裹着的舒高身躯,流畅的红发散至肩背;两只手撑着课桌,好奇地打量着,却无冒犯的意味。

“我叫吉田更纱,叫我更纱就好了!”更纱迫切的样子,似乎期望着斯提娅开口,想知道这位新同学,下一句会说什么。

“……叫我……斯提娅……好了。”斯提娅不知道如何表达。只是嚅嗫着断续着说出了这句再简单不过的介绍。随后,她低下了头,避开了那真诚的笑容和明亮的眸子。

“诶,”吉田一愣,正想说什么,却被嘉奈老师打断了,“好的,那么现在开始第一节课。”嘉奈老师顿了一顿,“以后课程分为三种,异能开发,理论综合,文化选讲,而今天的第一节课,是异能开发。”

嘉奈老师环视四周,看着一个个迷茫的同学,嘴角扯出一丝略狰狞的弧度,补充道:“所谓异能开发,也就是你们所说的实战课。”

“什么?一开始就实战?”“在搞什么啊!”“不应该是先讲讲理论知识吗?”教室里瞬间沸腾了起来。同学们都纷纷议论起来,就连斯缇娅的脸上也露出了一丝怪异。

“安静!”嘉奈老师重重地敲了敲黑板,“你们可以将这个当成一次开学前摸底考,如果有人接受不了,自然可以选择离开。同时,在考试途中如果生命垂危的话,你们也可以选择离开的哦!”

“什么?还会有生命危险?!”一些人面面厮觑了起来,面色开始有些发白,似乎有些退却。

“是的哦!”嘉奈老师露出了耐人寻味的笑容,“你们的资源分配将由这次考试的成绩决定,越晚被淘汰的人将受到越好的待遇,至于最先被淘汰的10个人,就只能被退学了哦!”

“那么怎么样算淘汰呢?”坐在教室右侧的一个男生举起了手。

“good question!”嘉奈老师笑了,“那么就姑且让我来介绍一下游戏规则吧!”

只见她手一挥,空中出现了一张图。“这是游戏场地的地图,各位将会被传送到随机的位置,进行为期五个小时的游戏,届时将统计每一个人手中淘汰的人的数量,并且排出名次,如果数量相等,按照被淘汰的先后来定。”好似故意卖个关子一般的停了一会,嘉奈继续说道:“至于淘汰吗,非常简单,你只需要使对方无法行动,就算你成功淘汰了他。当然,为了人道主义,我们不建议你杀了他。”

嘉奈想了想,又补充道:“对了,因为你们可能还比较弱,无法光用异能和拳头击败别人,所以场地内还会出现一些武器,当然仅限这次,你们可以随便使用。”

显然同学们又都注意到了仅限这次这句话,教室里又重新充斥起了讨论声。斯缇娅却是失望的摇了摇头,显然这所学校还是以培养战斗人才为主,所以异能更有用的地方还是战争吗,还是破坏吗……仿佛有些东西不由自主地浮现了出来,她的眼中透出一丝夹着悲伤的迷茫。

“那么,游戏马上开始,再给大家十五分钟的时间做准备,你们大可以在这段时间里组队。不过每队至多三人。”嘉奈老师随意地说着:“毕竟胜利才是最终目的,你们可以随意发挥。”

“那么各位,十五分钟后演习场见喽!”,嘉奈老师说完,哼着轻快的小曲离开了。

“呐,要不我们一起吧?”老师刚一走,吉田就跑了过来。斯缇娅抬起头,又对上了那对清澈的眸子,本来下意识脱口而出的“不用了”却被硬生生地卡了回去。在那段漂泊的岁月中,她早已习惯了一人独行,不是因为没人对她释放出善意,而是因为她早已不敢相信任何人。在那个异能被唾弃的时候,在那个异能被唾弃的地方,人们甚至会为了私人恩怨就举报对方是异能者,更何况本身是伪异能者的她呢?她早已习惯了独自在黑暗与光明的夹缝中生存,是的,不需要正义,也并没有什么值得批判,她早已麻木。也许这次可能是个改变,对方的目光很澄澈,很美好,就像夹缝外投入的足以驱散一切阴霾的阳光。但她仍然害怕,害怕这只是如同水中月镜中花一样,转瞬即逝,接着再一次回转回过往,阴霾密布,不见天日。没有的失望远远比拥有却失去的绝望要好得多。

“不用了……谢谢……”她叹了口气,回答道。

\emptypar

很难判断这里是哪里,斯缇娅只知道如此高大茂密的森林是如此少见,以至于足以使人产生本能上的恐惧。尽管这里阴暗得压抑,但透过树隙的硬直光斑,却显示树冠之上,不是常见的灰霾。斯缇娅回过头看了看倾斜着的终端,像是一头扎在交错的树根中,却毫无违和感。那是一个铝白色的长方体,正如一个移动厕所,表面却布满了锈迹,仿佛是上个纪元的冒险者的遗留物。正是这个不起眼的东西,却是嘉奈老师反复强调的邦联科技复兴的尖端,传送器的终端。

斯缇娅对这次突如其来的考试,确切地说,一点感觉也没有。她非常清楚,事情并不会沿着预期发展,只是吉田的突然邀请略微使她惊讶。不过这奇怪森林的压迫感还是促使着她快步行进。北原可不是这样,你只要站在市镇的大教堂的钟塔上,便可以遍览拥挤而矮小的平民的屋子,看到东边富人的花园华丽的双层楼房,目光顺着浅浅的水沟横穿城邦,直到触碰到灰青的荒草地,干燥土壤上生长着的三寸野草;再远处忽然出现蜿蜒的城墙,小得就像积木般,却是全邦联最不易被攻破的城墙,或许还是修补最多的城墙;奇怪的是城墙内外却没有什么区别,仍是灰青的草地,延伸到天边的灰霾中去;人们把这种望不见边际的荒草地叫做“灰原”,在北方灰原上建立的城邦便叫做“北原”。

斯缇娅不知道全邦联有那个城邦有想这样的茂密森林,不过她愿意相信这片奇怪的令人不安的森林是独一无二的。斯缇娅尽量快地走着,随之带来的落树叶的碎裂声使她更为不安。仿佛被人注视着,斯缇娅有这样的感觉。

等等,刚才是不是有什么……

斯缇娅猛地一回头,在光斑照不到的阴暗之中,倚靠在树干上,是一只曜黑的箭袋,冒出的箭尾闪着像铝一般的清澈光泽,还有耸直的塑料尾羽。斯缇娅犹豫了,这个位置太明显了,明显得让人怀疑是某人的把戏。但她最后还是小心地走上去,试探着抽出一支箭。晶亮的箭头甚至可以穿透目光。斯缇娅只觉得,这只箭太漂亮了,精致得不像战斗消耗品,而更像是工艺品。

斯缇娅把箭放了回去。数了数,总共七支箭。斯缇娅暗暗苦笑,真是个不吉利的数字。斯缇娅从古书上读到过,七是黄金纪元人们所相信的幸运数字。只不过七城之乱之后,七的吉利意味就同其中六个城邦一样,被无情的埋灭。只留下孤独的北原,孑立在邦联的北境,背负着过去的一切。

斯缇娅提起箭袋的肩带,拿起了压在下面的灰黑色的金属弓。两者加起来并不轻,斯缇娅却觉得非常踏实。至少,对于她这样的弱者而言,拥有了手中的寸铁。
\end{document}
