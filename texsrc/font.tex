\documentclass[a5paper, 10pt]{article}
\usepackage[scheme=plain, punct=kaiming, fontset=founder]{ctex}
\usepackage{blindtext}
\usepackage{metalogo}
\usepackage{biolinum}
\usepackage{parskip}
\usepackage{bookmark}
\usepackage{fancyhdr}
\usepackage{ifxetex}
\renewcommand{\familydefault}{\sfdefault}
\ifxetex\rfoot{\tiny\itshape Gen by \XeLaTeX{} with \CTeX{}}\fi
\lhead{\sffamily\rightmark}
\rhead{\leftmark}
\pagestyle{fancy}
\begin{document}
    \section{Biolinum}{
        Biolinum实在太好看了!我似乎对\textbf{喇叭口}情有独钟。
        真的是这种令人神魂颠倒的感觉。
        Biolinum不失为Optima的替代品。

        \subsection*{The quick brown fox jumps over the lazy dog}{
            \blindtext
        }
    }
    \pagebreak
    \section{语文书(苏教版)}{
        经过我的认真研究发现(雾),语文书用的是\textbf{开明式}的标点压缩。
        就是这样的效果:

        \subsection*{赤壁赋}{
            \kaishu
            苏轼

            \songti
            壬戌之秋,七月既望,苏子与客泛舟,游于赤壁之下。清风徐来,水波不兴。举酒属客,诵明月之诗,歌窈窕之章。少焉,月出于东山之上,徘徊于斗牛之间。白露横江,水光接天。纵一苇之所如,凌万顷之茫然。浩浩乎如冯虚御风,而不知其所止;飘飘乎如遗世独立,羽化而登仙。

            于是饮酒乐甚,扣舷而歌之。歌曰:“桂棹兮兰桨,击空明兮溯流光。渺渺兮予怀,望美人兮天一方。”客有吹洞箫者,倚歌而和之,其声呜呜然,如怨如慕,如泣如诉。余音袅袅,不绝如缕。舞幽壑之潜蛟,泣孤舟之嫠妇。

            苏子愀然,正襟危坐,而问客曰:“何为其然也?”客曰:“‘月明星稀,乌鹊南飞’,此非曹孟德之诗乎?西望夏口,东望武昌,山川相缪,郁乎苍苍,此非孟德之困于周郎者乎?方其破荆州,下江陵,顺流而东也,舳舻千里,旌旗蔽空,酾酒临江,横槊赋诗,固一世之雄也,而今安在哉?况吾与子渔樵于江渚之上,侣鱼虾而友麋鹿。驾一叶之扁舟,举匏樽以相属。寄蜉蝣于天地,渺沧海之一粟。哀吾生之须臾,羡长江之无穷。挟飞仙以遨游,抱明月而长终。知不可乎骤得,托遗响于悲风。”

            苏子曰:“客亦知夫水与月乎?逝者如斯,而未尝往也;盈虚者如彼,而卒莫消长也。盖将自其变者而观之,则天地曾不能以一瞬;自其不变者而观之,则物与我皆无尽也,而又何羡乎?且夫天地之间,物各有主,苟非吾之所有,虽一毫而莫取,惟江上之情风,与山间之明月,耳得之而为声,目遇之而成色,取之无禁,用之不竭,是造物者之无尽藏也,而吾与子之所共食。”

            客喜而笑,洗盏更酌。肴核既尽,杯盘狼藉。相与枕藉乎舟中,不知东方之既白。
        }

        \vspace{5mm}
        懒得设置行距,凑合着看吧。

        为什么我要把全文写出来呢?因为这篇是要默写的。
    }
    \pagebreak
    \section{字库}{
        一直以来我都使用Fandol字库来写中文。
        我很喜欢Fandol Hei的喇叭口设计,
        但Fandol系字体都只有两个字重,
        而且Fandol Hei Regular显得太粗,
        难以用来当正文字体(标题字体却比较合适)。
        而无衬线正文字体是低分屏屏幕阅读所必须的(机房垃圾屏幕)。
        于是在本文中,我尝试使用了方正字库。
        方正细黑(就是你现在看到的这种字体)的字重很适合正文屏幕阅读。
        出乎我的意料,字体质量还不错。(或许是我眼瞎)

        wps for linux自带方正字库,赞!

        至于日文字体,我以前一直用texlive带的
        IPA字库。不过它的假名连笔很严重(没有贬义)。
        或许Source Han JP(即Noto CJK JP)可能更符合我的审美。
        (有喇叭口!)
    }
\end{document}
